%%------------%%
%%-Eviroments-%%
%%------------%%

%%-----------%%
%%-Commmands-%%
%%-----------%%
% \hsp      % horisontal space
% \info     % info text, skal ikke være med i fullført kode
% \ebox     % Lager en svart box rundt parameteret sitt
% \tebox    % Tar inn 3 parametre, 1 størrelse, 2 plassering, 3 caption/label, setter bildet inn
% \kbox     % Tar inn samme 3 parametrene som tebox, legger de inn i ebox
% \mabox     % Tar inn ett eksra parameter, lager tebox, siste parameter er label.

\newcommand{\usikker}[1]{\textcolor{Peach}{#1}}

%Husker ikke kor vi bruker \hsp men vet vi bruker det en plass
\newcommand{\hsp}{\hspace{20pt}}

\newcommand{\info}[1]{
%   \textcolor{NavyBlue}{#1}
}

%Lager en svart box rundt parameteret sitt
\newcommand{\ebox}[1]{
    \setlength{\fboxsep}{1pt}   % 1
    \setlength{\fboxrule}{2pt}  % 2
    \fbox{#1}
}

\newcommand{\tebox}[3]{
\begin{figure}[ht]
        \begin{center}
        \includegraphics[width=#1\textwidth]{Images/#2}
        \end{center}
        \caption{#3}
        \label{#3}
    \end{figure}
}

\newcommand{\mabox}[4]{
\begin{figure}[ht]
        \begin{center}
        \includegraphics[width=#1\textwidth]{Images/#2}
        \end{center}
        \caption{#3}
        \label{#4}
    \end{figure}
}



%Tar inn 3 parameter og lager en figur
%parameter 1 = størrelse 1=full, 0,1 = 10% størrelse osv
%parameter 2 = filnavn på bildet
%parameter 3 = label og caption på bildet
\newcommand{\kbox}[3]{
\begin{figure}[ht]
        \begin{center}
        \ebox{\includegraphics[width=#1\textwidth-6pt]{Images/#2}}
        \end{center}
        \caption{#3}
            \label{#3}
    \end{figure}
}

\newenvironment{material}[2]{
    \begin{minipage}{0.3\textwidth}
    \ebox{\includegraphics[width=1\textwidth-6pt]{Images/#1}}
    \vspace{-20pt}
    \captionsetup{type=figure}
    \caption{#2}
    \end{minipage}\hfill
    \begin{minipage}{0.65\textwidth}
    \vspace{-30pt}
    \begin{itemize}
}
{
    \end{itemize}
    \end{minipage}
    \vspace{10pt}
}

\newcommand{\begrep}[2]{
\begin{minipage}{0.25\textwidth}
\textbf{#1}\end{minipage}
\begin{minipage}{0.65\textwidth}
#2
\end{minipage}\par
\vspace{-5pt}
}

\newcommand{\nomenom}[2]{
\begin{minipage}{0.13\textwidth}
#1\end{minipage}\begin{minipage}{0.05\textwidth}
-\end{minipage}
\begin{minipage}{0.77\textwidth}
#2
\end{minipage}\par
\vspace{-5pt}
}

\newcommand{\svar}[1]{
\underline{\underline{#1}}
}

\newcommand{\navn}[1]{
\medskip
\textbf{#1}\par
}

\newcommand{\toBilder}[6]{
    \begin{minipage}{0.5\textwidth}
    \includegraphics[width=1\textwidth-6pt]{Images/#1}
    %\vspace{-10pt}
    \captionsetup{type=figure}
    \caption{#2}
    \label{#3}
    \end{minipage}\hspace*{\fill}\begin{minipage}{0.5\textwidth}
    \includegraphics[width=1\textwidth-6pt]{Images/#4}
    %\vspace{-10pt}
    \captionsetup{type=figure}
    \caption{#5}
    \label{#6}
    \end{minipage}
}

\newcommand{\HRule}{\rule{0.7\linewidth}{0.5mm}}


\newcounter{teAppendixCounter}
%\renewcommand{\theteAppendixCounter}{\Alph{teAppendixCounter}} % Kommenter ut for å få tall istedet for A B C

\newcommand{\teAppendix}[2]{
    \refstepcounter{teAppendixCounter}\label{app:#2} 
    \begin{tabularx}{\textwidth}{p{.2\textwidth} p{.8\textwidth}}
    Appendix \theteAppendixCounter & #1 
    \end{tabularx}
    \par
    \vspace*{-5pt}
}

\definecolor{codegray}{gray}{0.9}
\newcommand{\code}[1]{
    \colorbox{codegray}{\texttt{#1}}
}

\newcommand{\doublesignature}[4]{
\vspace{1.5cm}
\noindent
\begin{tabular}{lcl}
    \makebox[5cm]{\dotfill} & \hspace{2cm} & \makebox[5cm]{\dotfill} \\
    #1 & & #3\\
    \textit{#2} & & \textit{#4}
\end{tabular}
\vspace{1cm}
}
\usepackage[utf8]{inputenc}
\usepackage[pdftex]{graphicx}
\usepackage[lmargin=25mm,rmargin=25mm,tmargin=27mm,bmargin=30mm]{geometry}
\usepackage[T1]{fontenc} % Må ha for å få "accent" letters, slik som æøå, greit å ha lell om vi skriver på engelsk
\usepackage{lmodern}

\usepackage[usenames,dvipsnames]{color} % Farge pakken vi bruker inneholder mange farger
%\usepackage{svg}        % for å ha svg filer i rapporten

\usepackage[font=small,labelfont=bf]{caption} % Fra NTNU mal
\usepackage{fancyhdr}
\usepackage{titlesec}
\usepackage{setspace} % Provides support for setting the spacing between lines in a document.
\usepackage{float}
\usepackage{lipsum}
\usepackage{url}
%\usepackage{natbib}
\usepackage[backend=bibtex]{biblatex}

\usepackage{textcomp}
\usepackage{gensymb} % gir muligheten til å skrive symboler slik som grader ( \degree ) 

%%%%%%%%%%%%%%%%%%%%%
% Math
\usepackage{amssymb}
\usepackage{mathrsfs}
\usepackage{amsthm}
\usepackage{amsmath}
%%%%%%%%%%%%%%%%%%%%%

%\usepackage{times}
%\usepackage{fourier} % Use the Adobe Utopia font for the document - comment this line to return to the LaTeX default

%%%%%%%%%%%%%%%%%%%%%
% Table
\usepackage{rotating}
\usepackage{booktabs}
\usepackage{makecell}                   
%%%%%%%%%%%%%%%%%%%%%

\usepackage{microtype}                  % Skriv litt smart angående marger, altså fyller ut siden bedre
\usepackage{wrapfig}                    % Brukt for å få figurer inn i teksten
\usepackage{pdfpages}                   % For å få inn pdf filer i latex
\usepackage[toc,page]{appendix}
\usepackage{adjustbox}
\usepackage{enumerate}                  % Advanzed Enumerate
\usepackage{subcaption}                 % Brukt for å få to figurer på samme linje
\usepackage[english]{babel}             % English language/hyphenation
\usepackage{glossaries}                 % For å lage sorterte lister (alfabetisk)
\usepackage{xargs}                      % Use more than one optional parameter in a new
\usepackage{parskip}

%%%%%%%%%%%%%%%%%%%%%%%%%%%%%%%%%%
%%%% Fra tidligere prosjekter %%%%
%%%%%%%%%%%%%%%%%%%%%%%%%%%%%%%%%%
%
% Look & Feel
% %%%%%%%%%%%%%%%%%%%%%%%%
% % Coding
%     \usepackage[framed]{mcode}              % Matlab greie
      \usepackage{listings}                   % For å poste code inne i latex dokumenter
\lstset{ %
language=C,                     % choose the language of the code
basicstyle=\footnotesize,       % the size of the fonts that are used for the code
backgroundcolor=\color{white},  % choose the background color. You must add \usepackage{color}
showspaces=false,               % show spaces adding particular underscores
showstringspaces=false,         % underline spaces within strings
showtabs=false,                 % show tabs within strings adding particular underscores
frame=single,                   % adds a frame around the code
tabsize=2,                      % sets default tabsize to 2 spaces
captionpos=b,                   % sets the caption-position to bottom
breaklines=true,                % sets automatic line breaking
breakatwhitespace=false,        % sets if automatic breaks should only happen at whitespace
escapeinside={\%*}{*)}          % if you want to add a comment within your code
}
% %%%%%%%%%%%%%%%%%%%%%%%%
% 
% %%%%%%%%%%%%%%%%%%%%%%%%
% % Matte
%     \usepackage{mathtools}
%     \usepackage{cancel}                     % cancel ut ting i matte
%     \usepackage{amsmath,amsfonts,amsthm}    % Math packages
% %%%%%%%%%%%%%%%%%%%%%%%%
%    
% \usepackage{pdflscape}
% \usepackage{lscape}                         % For å få landskapsmodus på en side
%   
% \usepackage{multirow}
% \usepackage[table,xcdraw]{xcolor}
% \usepackage{rotating}
% \usepackage{booktabs}
% \usepackage[normalem]{ulem}
%
% \usepackage[sort, numbers]{natbib}
%
% \usepackage{array}
% \usepackage{tabularx}
%   
%
% % Hyper refs, måtte være siste pakke ifølge foreleser om latex
% \usepackage[colorlinks=true, a4paper=true, pdfstartview=FitV, linkcolor=blue, citecolor=blue, urlcolor=blue]{hyperref}
%
% \usepackage{attachfile}
\usepackage[pdftex,bookmarks=true]{hyperref}

\usepackage[pdftex]{hyperref}
\hypersetup{
    colorlinks,%
    citecolor=black,%
    filecolor=black,%
    linkcolor=black,%
    urlcolor=black
}
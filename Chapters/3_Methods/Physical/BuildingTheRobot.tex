\subsection{Designing the robot}
The robot design were planned out during the first week of the project. During this week the group discussed which parts where necessary to build a robot that meets our goal. The entire robot was planned out before ordering any parts to make sure that there was no unnecessary purchases. It was important to order all parts as soon as possible as shipping time can be large when ordering from across the planet.

The robots should be enable to charge itself when it has low battery, and there is currently one charger that every robot shares. This means that the new robot should be able to use the old charger. Because of this, the new robot has a 11.1V lithium battery which can charge itself using the same charger. This was bought from a Norwegian vendor called Elfa Distrelec\cite{elfa} since the price of lithium batteries was the same as ordering from outside Norway.

Erlend Ese bought five nRF51 Dongles from Omega Verksted NTNU, one of these where used in the robot.

All metallic parts and the two charging towers where manufactured by the cybernetics mechanical workshop at NTNU.

\improvement{Putte dette i resultat?}{The \acrshort{pcb} where designed in Eagle and printed by Elprolab at NTNU\cite{elprolab}. The plastic housing around the motors where designed in SolidWorks and 3D printed by the cybernetics mechanical workshop.}

All remaining parts where ordered from SparkFun early in the second week.

%%%%


\subsubsection{Assembly}
The robot chassis consists of two parts which can be separated into the top and bottom.

bottom

The battery, wheels and IMU is all mounted at the bottom part. To make the car turn around its own axis, the best placement for the two wheels was in the exact middle of the car. The IMU is placed right between the two wheels. 
\task{Skrive om kordan bilen er bygd opp}

\subsubsection{Wiring}
Skrive hvordan delene er koblet sammen, forklare koblingsskjema
\mabox{1}{Results/ardubot.pdf}{Arduino bot wiring}{fig:ardubot}

\subsubsection{Soldering}
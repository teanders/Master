\subsection{Improvements to the robot}
\label{sec:improvements}

\subsubsection{Designing the Motor Holder}
The assembled robot, as can be seen in Figure \ref{fig:robpremothold}, was too heavy in the center, and this led to the wheels leaning towards the chassis. As this would have influenced the performance of the robot while driving, the group decided to improve the assembled robot to neutralize the maldistribution of weight. It was decided to design and 3D-print a motor holder that should prevent the wheel angulation.

As SolidWorks, see Section \ref{sec:software}, is a modelling software, it was suited to use for this task. Measuring the physical robot gave an accurate basis for designing the motor holder, taking screws and nuts mounted to the chassis into account. The measurements and shapes obtained from the physical robot were modelled in SolidWorks, using one of the ``Get started'' tutorials in the software as guidance.

\textbf{Modelling the Motor Holder using Solidworks}
\task{Skal fremgangsmåten for modelleringen beskrives enkelt og greit her (det er sikkert greit sidefyll + enkelt å skrive om)? Hvis ikke, må de skrives noen setninger til som gjør at vi kan referere til bildene (\ref{fig:mothold1} og \ref{fig:mothold2})}

\mabox{.5}{Methods/RobotWithoutMotorHolder.png}{Robot without improvements}{fig:robpremothold}

\toBilder{Methods/MotorHolder1}{Motor holder design}{fig:mothold1}{Methods/MotorHolder2}{Motor holder design}{fig:mothold2}
\improvement{Hva skjer med caption på bilde \ref{fig:mothold1} og \ref{fig:mothold2} ?}

\subsubsection{Designing the Printed Circuit Board}
skrive om hvordan det ble designet/laget/tegnet
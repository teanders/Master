\subsection{The Gestalt Principles}
\label{sec:gestaltprinciples}
The Gestalt Principles is a framework that describes how users experience and perceive visual interaction with a user interface. Humans are ``programmed'' in a way that they should perceive their surroundings as whole objects, and they search for structure and contexts instead of accepting discontinuous lines and figures. There are a lot of Gestalt Principles, but the most relevant ones are \cite{johnson}:
\begin{enumerate}
     \item Proximity
     \item Similarity
     \item Continuity
     \item Figure/Ground
 \end{enumerate}

\subsubsection{Proximity}
\label{sec:proximity}
Distance between objects influences the users perception of whether the objects are organized in groups or not. Objects close to each other are perceived as connected, and thus are influenced by each other. The Proximity principle affects a \acrfull{gui} in a way that if the \acrshort{gui} is separated in groups the user will easier understand the system \cite{johnson}.

\subsubsection{Similarity}
\label{sec:similarity}
Similarity between objects causes the object to feel grouped, and thus can objects that are not in close proximity of each other still be perceived as having relevance to each other \cite{johnson}.

\subsubsection{Continuity}
\label{sec:continuity}
The human mind is ``programmed'' to see continuity and hence it fills in information to perceive discontinuous shapes as continuous \cite{johnson}.

\subsubsection{Figure/Ground}
\label{sec:figureground}
The visual system structures information and thus perceives that objects in the foreground has a bigger importance than objects in the background. Smaller objects placed on a bigger objects are perceived as a figure while the bigger objects are perceived as the ground beneath the figure. Using the principle of figure/ground can enable the \acrshort{gui} to present important information in the foreground \cite{johnson}.
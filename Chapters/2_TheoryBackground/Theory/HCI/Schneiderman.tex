%Grammarly 98
\subsection{Schneidermans 8 Golden Rules}
\label{sec:schneiderman}
The eight golden rules of Schneiderman are guidelines for \acrshort{gui} development. The principles were developed based on experience gained by perceptual and cognitive psychology in an attempt to improve the design of interactive systems. According to Schneiderman \cite{schneiderman}, the guidelines needs to be validated and adapted to specific design domains. A list like this will never be complete, but it has been received as a useful guide for students and designers.

\subsubsection{1. Strive for consistency}
Actions should be consistent in similar scenarios. This means that the colour, layout and font selection should be consistent, menus and help screens should use the same terminology, and so on \cite{schneiderman}.

\subsubsection{2. Cater to universal usability}
Frequent users want to minimise the number of interactions to increase the pace of interaction. Abbreviations, function keys and hidden commands suits the expert user, while explanations and features for novices suits the inexperienced user \cite{schneiderman}.

\subsubsection{3. Offer informative feedback}
Every user action should provide a system feedback. Frequent and minor actions should provide a modest response while infrequent and major actions should provide a more substantial response \cite{schneiderman}.

\subsubsection{4. Design dialogue to yield closure}
Action sequences should have a beginning, a middle, and an end. At the end of the sequence, the system should provide informative feedback to give the operator the satisfaction of accomplishment \cite{schneiderman}.

\subsubsection{5. Prevent errors}
The system should be designed in a way that the user is unable to make a serious error. If the user has made an error, the system should offer an easy instruction for recovery \cite{schneiderman}.

\subsubsection{6. Permit easy reversal of actions}
Actions done by the user should be reversible. Providing this feature relieves anxiety since the user knows errors can be undone \cite{schneiderman}.

\subsubsection{7. Support internal locus of control}
Experienced users should experience that they are in charge of the interface. They want the interface to respond to their actions without surprises. If the user is forced to fill in large amounts of data, has difficulties to get the required information, and has difficulties to perform actions, his satisfaction will decrease and stress increase \cite{schneiderman}
\newpage
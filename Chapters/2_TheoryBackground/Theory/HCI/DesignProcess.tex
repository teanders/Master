\subsection{Design Process}
\label{sec:designprocess}
In the process of making an artefact, four main steps are defined (see Figure \ref{fig:designprocess}) :
\begin{enumerate}
 \item ``Sense gap'': Design begins with a gap in the user experience. Without a gap in the user experience there is no need for design.
 \item ``Define problem'': Further, the problem with the gap has to be explained. In this stage of the design process the user needs has to be identified, and in the identification phase there is a possibility that the basis for the design process becomes wrong. The user is often not certain what he really wants, and what he explains can be perceived wrongly by the designer. It is important that the ``perfect system'' is described as thoroughly as possible.
 \item ``Explore alternatives'': In this step the designer tries to explore several alternative ways to find  different solutions to the problem. Involving the user in user testing, and to contact people that have an understanding of the problem, can give an idea of the right approach, an thus can the designer provide multiple solutions.
     \item ``Select plan'': The exploration usually provides several solutions to the problem, and therefore the plan selection is an important stage of the design process. Through conversations with the user the designer can narrow down the results of the previous phase to the best solution.
 \end{enumerate}
 It's usual to iterate through steps 1-3 several times before the best solution is found (\cite{ulrich}).
 
 \mabox{0.8}{Theory/designprocess.jpg}{The four main steps in a design process (\cite{ulrich})}{fig:designprocess}
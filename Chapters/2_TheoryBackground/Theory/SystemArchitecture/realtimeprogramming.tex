\subsection{Concurrent Programming}
\begin{itemize}
    \item Threading
    \item Shared Objects
    \item Synchronized code blocks
    \item Deadlocks
\end{itemize}

A concurrent program consists of several streams of operations that may execute concurrently. The program is built in a way that the streams executes at the same time and they can communicate with one another (\cite{cartwright2000}).

\subsubsection{Threading}
\label{sec:threading}
Each stream of instruction in a program is called a \textit{Thread}. If a multi-threaded application is running, one of the threads may override the other threads in a way that blocks the others. This is called \textit{starving}. To prevent starving (the Java language does not guarantee this), most \acrshort{jvm} provides fairness allowing all of the threads to execute their tasks (\cite{cartwright2000}).

\subsubsection{Shared Objects}
\label{sec:sharedobj}
Objects and variables that are accessed by several threads are called ``Shared objects/variables''. The use of shared object is a nice way to make threads interact with on another, but it is also a way to introduce concurrency problems to the application. Some of the problems and solutions are further explained beneath (\cite{downey2008}).

\textbf{Race Conditions}
\label{sec:raisedcond}
Race conditions 

\textbf{Synchronization}
\label{sec:synchronization}

\textbf{Deadlocks}
\label{sec:deadlock}
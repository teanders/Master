\subsection{Micro computers}
\subsubsection{Arduino Mega 2560}
There's many different versions of Arduino boards, providing different physical sizes, processors and number of \acrfull{io} \cite{arduinoboards}. 

The Mega 2560 is an improved update to the Arduino Mega. It is has 54 digital \acrshort{io} pins and 16 analog input pins, it is based on the ATmega2560. Figure \ref{fig:atmegaPinMap} illustrates the mapping between the Arduino pins and ATmega2560 ports \cite{arduinomega2560}. 

\mabox{0.8}{Theory/atmega2560.png}{Arduino Mega 2560 - ATmega2560 pinmap}{fig:atmegaPinMap}

\task{Pga layouten til Megan er det greit å lage et egendefinert skjold til systemet}

\subsubsection{nRF51}
The nRF51 series supports several different protocol stacks including Bluetooth Smart (Bluetooth low energy), ANT and proprietary 2.4GHz protocols such as Gazell. 

The nRF51 Dongle is a low-cost USB development dongle, it has 6 \acrfull{gpio} pins and it can be used as a peripherial between computers and robots \cite{nrf51Dongle}.

\mabox{0.2}{Material/Nrf51dongle.JPG}{nRF51 dongle}{fig:nrf51Dongle}

\unsure{Usikker på om vi skal ha bilde her eller ikke}{}
%Grammarly
\subsection{Micro computers}
\subsubsection{Arduino Mega 2560}
There's many different versions of Arduino boards, providing different physical sizes, processors and number of \acrfull{io} \cite{arduinoboards}. 

The Mega 2560 is based on the ATmega2560 and it is an improved update to the Arduino Mega. It provides 54 digital \acrshort{io} pins and 16 analogue input pins. Figure \ref{fig:atmegaPinMap} illustrates the mapping between the Arduino pins and ATmega2560 ports \cite{arduinomega2560}. 

\mabox{0.75}{Theory/atmega2560.png}{Arduino Mega 2560 - ATmega2560 pinmap}{fig:atmegaPinMap}

\subsubsection{nRF51}
The nRF51 series supports several different protocol stacks including Bluetooth Smart (Bluetooth low energy), ANT and proprietary 2.4GHz protocols. 

The nRF51 Dongle is a low-cost USB development dongle; it has six \acrfull{gpio} pins and can be used as a peripheral between computers and robots \cite{nrf51Dongle}.

%\mabox{0.2}{Material/Nrf51dongle.JPG}{nRF51 dongle}{fig:nrf51Dongle}

%\unsure{Usikker på om vi skal ha bilde her eller ikke}{}
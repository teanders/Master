%Grammarly ok
\subsection{Electronics}
\subsubsection{Capacitor}
\label{sec:capacitor}
Capacitors can be found in different sizes and materials, the most popular being electrolytic capacitors for values greater than 1µF, and ceramic capacitors for lower values. In this thesis there were mainly two applications areas for capacitors, these are further explained below.

If a component has a sudden high demand for power, the demand can be enough to drop the voltage in the circuit and other components can start to misbehave. By adding a high-value capacitor between ground and VCC, the capacitor will act as a reservoir of electricity, so that the component will draw the charge from both the supply and the capacitor. A typical component where this is needed is a servo motor that draws a lot of power as the motor is starting. A good default value for a servo motor is 430uF, depending on the application \cite{highcapacitor}.

The capacitor also has another great feature; in a DC circuit, it can filter out high-frequency AC noise. Not all components need pure DC signal, but some components, such as logic chips, may start to operate incorrectly if the voltages swing to much. By placing a capacitor between ground and VCC, the capacitor acts as a bypass capacitor and shorts the AC signal which removes any AC noise on the DC voltage. Typically this kind of capacitor is smaller in size than the first described one; a good default value is 100nF ceramic capacitor \cite{lowcapacitor}.

For both type of capacitors, it is important to put it as close to the component as possible.

\subsubsection{Infrared Sensor}
\acrfull{ir} sensors work by emitting an infrared light and then detecting if the light gets reflected back to the sensor. There are two types of \acrshort{ir} sensors, one who only detects if an object is in close proximity and one that can determine how far away an object is. Since the sensor work by emitting and analysing infrared light, it can be affected by other infrared sources like sunlight \cite{infrared}. 

The sensors are an affordable alternative compared to other range sensors. They measure at a narrow beam width which can result in a more detailed representation of the environment \cite{infrared}.

The distance to an object is measured by triangulation. When light returns, it comes at an angle dependent on the distance of the reflecting object, which Figure \ref{fig:ir} illustrates. The distance can then be calculated from the angle. Light does not reflect the same way off every surface and the sensor reading will be different for different surfaces and/or colours even if the range is the same \cite{infrared}.

\mabox{0.5}{Theory/ir.jpg}{Infrared triangulation \cite{irjpg}}{fig:ir}
\newpage
\subsection{Java}

Java is an object-oriented programming language developed mainly by James Gosling and other developers at Sun Microsystems. The language is based on the ``WORA'' principle (\cite{wora}), ``Write Once, Run Anywhere''. This means that Java does not compile to machine code, but instead is compiled and run by \acrfull{jvm}. In this way one can run Java applications on all systems where there is a \acrshort{jvm} (\cite{java}).

All sourcecode is written in plain text using the ``.java'' file format. The files are compiled to ``.class'' format that contains bytecode which \acrshort{jvm} uses to run the application. The process is illustrated in Figure \ref{fig:jvm}.

\mabox{0.8}{Theory/JVM.png}{The process from sourcecode to compiled bytecode (\cite{jvm})}{fig:jvm}

The Java \acrfull{api} is a big collection of ready-made software components that are grouped in libraries that consists of classes and interfaces that are related to each other.

Java requires a software platform to run compiled applications, and one of the official platforms are ``Java SE'' (\cite{java}).
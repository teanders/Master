\subsection{Wireless Communication}
\subsubsection{Bluetooth}
\textit{Bluetooth} is a low energy, short-range technology created in 1994 \cite{ble2016}. It was developed by \acrshort{sig} and named after the viking king Harald Blåtand. The intention of Bluetooth was to allow wireless connectivity and collaboration between products \cite{bluetoothsig2016}.

Bluetooth technology ensures that devices are capable of communicating with each other regardless of the manufacturer of the device \cite{prabhu2004}.  

\textbf{Specifications}
\\
The different Bluetooth versions have different specifications, but as the newer versions are backwards compatible \cite{bluetoothreport2013} they cover the older versions. As \textit{Bluetooth Smart} (represents all Bluetooth versions from 4.0) is used in this thesis, specifications for that standard is listed below:

\begin{itemize}
	\item Theoretical maximum data rate is 260 kbps.
	\item Power consumption is $\sim$100 \micro Ah per day.
	\item In 3ms a device can connect, send and acknowledge data.
	\item Bluetooth Smart advertises on 2402, 2426 and 2480 MHz, and thus avoids interference with Wi-Fi traffic \cite{csr2010}.
\end{itemize}
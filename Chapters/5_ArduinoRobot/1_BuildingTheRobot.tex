\section{Designing the robot}
\label{sec:robotdesign}
The robot design was planned out during the first week of the project. During that week, the group discussed which parts were necessary to build a robot that met our goal. The entire robot was planned out before ordering any parts to make sure that there were no unnecessary purchases. It was important to order all parts as soon as possible as shipping time can be significant when ordering from the US.

\subsection{Acquiring materials}
The chassis and most of the electronic parts of the Arduino robot were ordered from SparkFun \cite{sparkfun} early in the second week. As the project progressed, some additional parts were needed and the group ordered metallic parts and the two charging towers at the Cybernetics Mechanical Workshop at NTNU.

The robot should be able to charge its battery when it is low and as there is only one charger that all of the robots shares, the robot should be able to use that charger. Because of this, the Arduino robot has an 11.1V lithium battery which suits the named charger. The battery was bought from a Norwegian vendor called Elfa Distrelec\cite{elfa} since the price of lithium batteries was the same as ordering from outside Norway.

Ese (2016) bought five nRF51 Dongles from Omega Verksted NTNU  for communication use, and one of these was mounted on the Arduino robot.

The \acrshort{pcb} were designed in Eagle, see Section \ref{sec:software}, and printed by Elprolab\cite{elprolab} at NTNU. The plastic housings around the motors were designed in SolidWorks and 3D printed by the cybernetics mechanical workshop.

The switches, \acrfull{led}, resistors, capacitors and wires were all acquired from the Cybernetics Electrical Workshop at NTNU.

%%%%

\subsection{Assembly}
The robot chassis consists of two parts that can be separated into top and bottom. Additionally, all around the robot small plastic pillars were installed to connect and support the two parts as well as the charging towers are designed to add an extra connection between the two.

\minisection{Bottom}
The driving wheels were initially mounted in the centre on the bottom plate; this makes the robot turn radius as small as possible. The \acrshort{imu} was placed between the two wheels at the exact centre of the robot, which makes the \acrshort{imu} output as precise as possible. The wheels and \acrshort{imu} where moved at a later stage of the project, this is further explained in Section \ref{sec:improvements}. The metal casting was mounted in the centre back of the car, and a thin metal plate was cut to make sure the casting ball was at the correct height so that the bottom plate were parallel to the ground. The battery, which is the heaviest component on the car, was placed right above the casting ball, making sure the robot is tail heavy. As there was only a small space left in the front of the bottom plate, the breadboard was placed there. Figure \ref{fig:bottomside} show the bottom part of the robot.

\minisection{Top}
The servo and the \acrshort{ir} tower was placed in such order that the tower rotates at the centre of the robot. The tower itself was made out of Lego bricks, the same way as the \acrshort{ir} tower on the other robots, and it is easy to take off and switch sensors if needed. The motor controller is placed right behind the \acrshort{ir} tower, and in front of the tower is the Arduino mounted. Next to the motor controller there are two switches, one for powering the robot and one for enabling the charging towers. A switch on the charging tower is something the other robots do not have, but the group decided that it was necessary since driving with the charging towers enabled is a hazard. E.g. the robot moves around and makes the charging tower hit something that can conduct electricity it will short-circuit the battery, but the switch will eliminate this danger. The switches are marked with clear symbols to reduce user failure. Additionally, there is installed a glass fuse to protect the Arduino. The motor controller, \acrshort{ir} tower and the Arduino is elevated so that wires can run beneath them. Connected to the top plate but facing towards the bottom plate is the wheel encoders. The encoders are placed in such order that they match up with the motors mounted on the bottom plate, with about 2mm distance from the magnet rings on the motors. Figure \ref{fig:topside} shows the bottom part of the robot.

There are several pictures and videos of the robot in Appendix \ref{app:media}.

\toBilder{Methods/RobotBottomside.png}{Bottom part of the robot}{fig:bottomside}{Methods/RobotTopside.png}{Top part of the robot}{fig:topside}

\subsection{Wiring}
Before the components were connected, a wiring diagram was created. The diagram was updated whenever an improvement on the design was made, and Figure \ref{fig:arduinomicrostation} shows the newest version. The diagram ensured that all connections were done correctly and made it easier for maintenance/troubleshooting. Standard colouring conventions were used, with red as positive power and black as ground while signal cables used other colours depending on the component. When the prototype first was created, all wires were connected to a breadboard. This made it easy to test if all components were correctly connected and if they worked as intended. At a later stage in the project, the breadboard was replaced by a PCB that acts as an Arduino shield, further explained in Section \ref{sec:improvements}.
\mabox{1}{Results/ardubot.pdf}{As-built wiring diagram for the Arduino robot}{fig:arduinomicrostation}

Connections between components were made with self-made jumper wires, which makes it more straightforward to troubleshoot than many single wires on the same board. Soldering was used to connect wires to the two switches and the charging tower springs. All the soldering work was done in the Cybernetics Electrical Workshop since the workshop provides a safe environment with sufficient ventilation.

As seen in Figure \ref{fig:arduinomicrostation} the 5V was taken from the voltage regulator on the motor controller and not from the Arduino, this because the voltage regulator on the Arduino could not supply enough power. The \acrshort{imu} and the six \acrshort{gpio} pins on the nRF51-dongle require 3.3V power, so these wires go via the Logic Level Converters. For communication, nRF51 uses \acrshort{uart} (Section \ref{sec:uart}) and the \acrshort{imu} uses \acrshort{spi} (Section \ref{sec:spi}).

As explained in Section \ref{sec:capacitor}, a capacitor is often used to prevent voltage level drop when components draw too much power. When the servo motor starts it demands a lot of power, so a capacitor is placed right next to it. Also, a ceramic capacitor is placed next to the nRF51-dongle to remove AC noise on the DC voltage. 
\chapter{Project Management}
\label{cha:projectmanagement}
Using Microsoft Project, see Section \ref{sec:software}, as project management tool has enabled the group to get useful insight in the project progress. Tools like MS Project lets the user plan every part of a project in detail, which is necessary and valuable both for small and large projects (e.g. a Master Thesis).

MS Project has provided information about how the project has progressed and has made it possible to adjust the workload according to planned progress. The adjustments provided a chance to stick with the schedule and made the group able to complete the project within the deadline.

Throughout the project, the team has logged all that has been done, and this has proven to be a useful resource during the writing of this thesis. The log contains how many hours each group member has worked, and which tasks the members has been working with at the given time. The log has contributed to a general insight in what the other member has been working at.

As well as logging the work, the group has held meetings with both the employer and the other teams that have been working on the other parts of the system. Doing so has enabled the teams to cooperate the best way possible, and made them able to finish the project in time. The group has logged the meetings in ``Minutes of Meeting'', see Appendix BLABLA. \task{appendiiix}

During the development of the software application has Github \unsure{Usikker på om dette skrives om}{,see section \ref{sec:software},}  been used for version control, and thus has enabled the groups to cooperate writing code on the same application. In addition to version control, the teams has used a common document in Microsoft OneNote to share thoughts and ideas throughout the project period.
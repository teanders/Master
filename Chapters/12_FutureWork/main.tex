\chapter{Future Work}
    \section{Create an Updated PCB Sheild}
        \label{sec:fwSkjold}
        As described in Section \ref{sec:finardu}, there were made some changes to the robot after the \acrshort{pcb} shield was printed. This resulted in that the group had to change some of the wires on the \acrshort{pcb} by manually soldering extra wires onto the \acrshort{pcb}. 

        Updating the Eagle files in Appendix \ref{app:wiring} with the changes done on the shield, then print a new \acrshort{pcb} and then solder the board is a good project that should be done. In Appendix \ref{app:wiring} there is also a wiring diagram of the robot as-built, the new Eagle schematic should be updated with the same connections.

    \section{Repair the Encoder}
        As described in Section \ref{sec:finardu}, the Arduino robot's wheel encoder got damaged during final testing; this is the only part that does not work on the robot. It may be possible to repair the encoder, if not they can be bought from SparkFun \cite{sparkfun}. The software code on the robots that reads the encoder should also be looked at as a better position estimate can be found if the robot uses compass, accelerometer and gyroscope in addition to the wheel encoders.

    \section{Communication}
        As described in Section \ref{sec:nrf51asCom} the nRF51-dongle is limited to maximum three connected robots. If it is desired to continue to scale this robot project with more robots, it is necessary to find a better solution than the current nRF51-server and nRF51-peripheral code.

    \section{Java Application}
        The Java server application is designed to be a foundation for future projects. If the communication gets changed in the future, only the communication package will be affected. Hence, only classes inside the communication package should be changed.

        The application is ready for future improvements like an improved Simulator, \acrshort{ai} or \acrshort{slam}.



    \section{Tenke}
\begin{itemize}
    \item nRF51:
    \begin{itemize}
        \item Forbedre kode
        \item Øke kommunikasjonshastighet (melding pr sekund)
    \end{itemize}
    \item Bytte trådløs protokoll fra bluetooth til noe annet (wifi?)
    \item Printe og lodde nytt kretskort
    \item Utvide funksjonalitet i robot info:
    \begin{itemize}
        \item Ha personlige kart for hver robot
        \item Mer fancy kartplotting?
        \item I tillegg til sensorinformasjon, kan det også lagres bilder til applikasjonen (fra webkamera/actionkamera)
    \end{itemize}
    \item Bytte fra IR til annen type sensor (lidar?)
    \item Legge til messagedeadline-funksjonalitet
\end{itemize}
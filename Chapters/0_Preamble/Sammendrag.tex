\chapter*{Sammendrag}
\textit{Prosjektets gjennomføring var delt inn i tre temaer. En Arduino-basert robot skulle bygges for bruk i kartlegging av ukjente områder. Videre skulle en serverapplikasjon utvikles med mulighet for å kontrollere flere roboter i sanntid, samt å kunne bruke informasjonen robotene samlet til å danne et felles kart av dataene. I tillegg skulle den trådløse kommunikasjonen fra den eksisterende løsningen oppdateres til å bruke nyere teknologi.}

\textit{En Arduino-robot ble designet og bygd ved å bruke materialer og deler fra SparkFun, Elfa Distrelec og de kybernetiske vekstedene ved NTNU. Roboten tilbyr den samme funksjonaliteten som de eksisterende robotene, men da den ene hjul-enkoderen ble ødelagt iløpet av den siste uken av prosjektet, fungerer ikke posisjonsestimering i roboten.}

\textit{Kommunikasjonsstandarden ble oppdatert fra Bluetooth til Bluetooth Smart, ved å bruke nRF51-dongler utviklet av Nordic Semiconductor. Programvaren som kjører på serverdongelen ble utviklet i programmeringsspråket C.}

\textit{Ved å studere den eksisterende serverapplikasjonen, skrevet i MATLAB, tilegnet gruppen seg en innsikt i systemets funksjonalitet. Mesteparten av den samme funksjonaliteten ble implementert i en ny serverapplikasjon utviklet i Java. Systemarkitekturen ble designet ved hjelp av moduler, og hver modul hadde hvert sitt unike ansvarsområde.}

\textit{Gruppen foretok en omfattende designprosess for å utvikle et grafisk brukergrensesnitt for systemet. Ved å bruke velkjente prinsipper og retningslinjer under utformingen, ble brukergrensesnittet designet med menneskets oppfattelsesevne i fokus. Brukergrensesnittet ble utviklet ved hjelp av en GUI-builder i utviklingsverktøyet NetBeans, og målet var å implementere all funksjonaliteten fra den eksisterende løsningen, men presentere denne på en mer brukervennlig måte.}
\section{Creating the Message Protocol}
When implementing the communication between the physical robots and the server software, the groups realized the need for a general message protocol. The message protocol should be complex enough to consist of different message types, such as updates and handshakes, and variables, such as parameters for physical size, as well as being simple enough to be easy to understand and implement in software.

The groups sat down and discussed which types of messages the protocol should implement, and what the message types should include of parameters. The message types and parameters were listed as shown in Table \ref{tab:messprotfromrob} and Table \ref{tab:messprotfromser}:

\begin{table}[ht]
\begin{center}
 \begin{tabular}{|l | l|} 
 \hline
 Message type       &	Parameters \\
 \hline
 \hline
 From Robot: & \\
 \hline
 Handshake          &   Message type \\
 					&	Physical robot width and length  \\
                    &   Tower offset from center of the robot\\
                    &	Axle offset lengthwise \\
                    &	Sensor offset from tower center \\
                    &	Initial sensor heading \\
 \hline
 Update  			&   Message type \\
                    &   Robot position (x,y) \\
                    &   Orientation \\
                    &   Tower heading \\
                    &	Sensor values \\
 \hline
 Status		        &   Message type \\
 					&	Destination arrived \\
                    &   Wall hit \\
 \hline
 \end{tabular}
\end{center}
\caption{Criterion for functionality in the \acrshort{gui}}
\label{tab:messprotfromrob}
\end{table}

\begin{table}[ht]
\begin{center}
 \begin{tabular}{|l | l|} 
 \hline
 Message type       &    Parameters \\
 \hline
 \hline
 From Server: & \\
 \hline
 Update  			&   Message type \\
                    &   Orientation \\
                    &   Distance \\
 \hline
 Status		        &   Hanshake confirmed \\
                    &   Pause robot \\
                    &   Unpause robot \\
                    &   Robot finished \\
 \hline
\end{tabular}
\end{center}
\caption{Criterion for functionality in the \acrshort{gui}}
\label{tab:messprotfromser}
\end{table}

At first the group developed a message protocol using JSON-structure. The benefit of using JSON to structure a message protocol, is that it is easy to understand just by reading the message as well as the order of the parameters does not have to be specified. The update message from server in JSON-structure was defined as:

\{``Update'':[\{``Orientation'':``60''\},\{``Distance'':``100''\}]\}

Due to some problems with the nRF51-dongle, see elaboration in Section \ref{sec:wirelesscomresult}, the group converted the message protocol from JSON-structure to a more compact designed structure. The update message was then defined as:

\{U,60,100\}

The remaining message types were defined in the same manner.
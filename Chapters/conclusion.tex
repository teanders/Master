During the Master Thesis, the group has developed a server application and a \acrlong{gui} in Java, software for an nRF51-dongle in C, as well as built an Arduino-robot. The different parts of the project seems to be good solutions to the problem description the group made at project start.

Choosing Java as programming platform facilitated a clear structure of the code and provided functionality for dividing the system into modules. The server application is developed in terms of the guidelines in ``Code Complete'', and is structured in a way that makes it easy for future projects to understand and further develop the software by adding new functionality. The code is generalised, so that it is possible to add robots with different specifications than the ones already existing, e.g. a robot with more than four infrared sensors. The server application inherits most of the functionality implemented in previous solutions, however, in our implementation the system is processing data and controlling the connected robots in real-time.

The user interface is developed during a design process where the human perception has been in focus. The process has helped the group focus on the aspects that are important when designing a graphical user interface, and the final product inherits most of the functionality provided in the previous solutions, but presents them in a more user-friendly way.

Choosing Bluetooth Smart as communication protocol enabled fast, short-range and low power consuming communication. The protocol was not known to the groups at beforehand, but by studying the standard, the groups managed to implement it into the system. The server-dongle receives and processes five messages per second from each robot connected, and delivers update and status messages to the robot dependent on the server application state. The communication range suited for the system is less than 15 meters between the server and the connected robots, this to ensure that all messages are received.

The group has built a new robot, increasing the existing collection of robots. The robot uses an Arduino as the microcomputer and it is built using the same wheels, gears, infrared sensors, tower and intertial measurement unit as the AVR-robot. Using AVR-dude enabled the Arduino-robot to use the same real-time operating system as running on the AVR-robot, without having to rewrite too much of the code. One of the wheel encoders got damaged during the last week of the project, and therefore it cannot estimate its position at this point in time. Other than the broken encoder, the robot works as intended and is therefore successfully built and implemented.

During the project period, the group has gained useful experience regarding project management and implementation. As ``\acrlong{ssnar}'' is a part of a system consisting of three dependent projects, the key to success has been communication and cooperation. Regular conversations and meetings with both the employer and the other groups working with the system have enabled effective solving of any changes to the project tasks. Using Git as version control tool for the developing of the software made it possible for several groups to work on the same application.

Compared to the scope of the project, the group is pleased with the technical result. The solution meets all the requirements of technical functionality and the solution seems to be a modernised solution compared to the prior implementations of the system. Updating the communication standard from Bluetooth to Bluetooth Smart, rebuilding the server application from MATLAB to Java, and building a new robot based on an Arduino microcomputer can all be said to extend the system in a provident way.
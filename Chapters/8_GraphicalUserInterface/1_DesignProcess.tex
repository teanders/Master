\section{Design Process}
\label{secr:designproc}
\textit{The design process was divided into 5 phases, and each phase dealt with the various key elements in a design process.}

\subsection{Sense the gap}
\label{sec:sensethegap}
To sense the gap, the group studied earlier projects to get an insight in the previous solutions to the problem. The group discovered several shortcomings in the existing solution and got an understanding of the gap in the user experience. The group draughted a list of criteria (see list below) that the final product should meet, and it laid the foundation for further process. The new design should:

\begin{itemize}
    \item Be reliable.
    \item Be easy to use.
    \item Minimise the likelihood of user error.
\end{itemize}

\subsection{Understand and refine the problem}
\label{sec:understprob}
Further, the group defined the problems of the existing solutions and its shortcomings. ``\acrlong{ssnar}'' is a project that has been performed several times, but the system is no longer working. This year's thesis main objective is to produce a better solution that will last significantly longer than the previous solutions. The group explored the existing solution to understand why it did not work as expected and tried to connect the findings to the three criterion defined in the ``Sense the gap'' phase.

\minisection{Reliability}
The existing solution was not perfect regarding reliability. During testing of the system the user often experienced that the actions not always led to the expected behaviour. Many of the button's functionality had been removed, but the buttons were still present in the \acrshort{gui}. With this in mind, the group designed an interface that consisted of a minimal set of functionality that met the requirements for functionality, as well as removing redundant components.

\minisection{Usability}
The existing solution had a \acrshort{gui} that contained lots of information and functionality. Many of the buttons were connected without any information about how they were related, some of the buttons were representing the same actions, and some buttons did not work at all. The buttons, together with many boxes presenting information, made the user interface complex and bad in terms of usability. The group used the previous solution to define guidelines for how the new system should be considering usability.

\minisection{Minimize the likelihood of user error}
Putting all of the functionality in the same window makes it more likely for the user to make mistakes. Dividing functionality into sections and making the program as a sequence of actions prevents user errors, and makes the system appear more usable and reliable for the user. This criterion laid the basis for this years solution. 

\subsection{Explore}
\label{sec:explore}
Based on the information in the specification, the exploration phase started. The group members developed various suggestions for the \acrshort{gui} with the specification as basis, but also took the Gestalt principles (section \ref{sec:gestaltprinciples}) and Schneiderman's 8 golden rules (section \ref{sec:schneiderman}) into account.

Before developing design solutions the group made a list consisting all of the functionality that the \acrshort{gui} should provide (see Table \ref{tab:guicriterias}).
\begin{table}[ht]
\begin{center}
 \begin{tabular}{|l l|} 
 \hline
 Maps               &   World \\
 \hline
 Robot information  &   Robot name \\
                    &   Position \\
                    &   Orientation \\
                    &   Tower angle \\
 \hline
 Indicators         &   Simulator on/off \\
                    &   Number of robots connected \\
 \hline
 Buttons            &   Simulator on/off \\
                    &   Manual drive/Autonomous \\
                    &   Connect to /disconnect robots \\
                    &   Start/Stop \\

\hline
\end{tabular}
\end{center}
\caption{Criterias for functionality in the \acrshort{gui}}
\label{tab:guicriterias}
\end{table}

\minisection{\acrlong{gui}, suggestion 1}
\mabox{.9}{Results/GUI1.PNG}{\acrlong{gui}, suggestion 1}{fig:gui1}
The suggestion in Figure \ref{fig:gui1}, has a start-up screen where the user selects either to run the simulator or to run the system in the real world. Choosing the mode brings the user to the main screen. In the main screen, the information about the robot is presented as well as a local/world map toggled by a radio button group. To add more robots the user has to click on the plus sign (new pane), and a pop-up window presents available robots.

\minisection{\acrlong{gui}, suggestion 2}
\mabox{.9}{Results/GUI2.PNG}{\acrlong{gui}, suggestion 2}{fig:gui2}
The proposal that is shown in Figure \ref{fig:gui2} has, like the first proposal, a start-up screen that presents two different modes (simulator and real world). Choosing the mode brings the user to a new dialogue box where the user is presented with another two options: manual drive or autonomous. Clicking on the desired mode brings the user to the main screen. The main screen consists of a horizontal list of the connected robots and their information, as well as a map that can switch between different modes. To add a robot the user has to click ``Add robot'', and he/she is then presented available robots in a pop-up.

\minisection{\acrlong{gui}, suggestion 3}
\mabox{.9}{Results/GUI3.PNG}{\acrlong{gui}, suggestion 3}{fig:gui3}
The suggestion that is shown in Figure \ref{fig:gui3} does not have a start-up screen. The main screen presents the two different modes in the top left corner, and the rest of the functionality will be available after selecting a mode and pressing start. Further, the main screen contains a list of connected robots and a world map. Clicking ``connect to more robots'' presents a pop-up similar to the other solutions while clicking ``more'' (buttons next to connected robots) opens the robot screen. The robot screen presents information about the specific robot as well as an opportunity to enter manual drive mode for the robot.

\subsection{Evaluation and feedback}
\label{sec:evalandfeedback}
To evaluate the different suggestions, the group printed the \acrshort{gui} suggestions on sheets and arranged usability testing to get user-experienced feedback. The different screens of the interface were covered using sticky notes, but these were removed as the test person ``navigated'' through the \acrshort{gui}. The test person was given some pre-defined tasks to perform, and during the interaction he/she was not receiving any help. After performing a set of tasks, the person was handed an evaluation sheet, where he/she should evaluate the different parts of the \acrshort{gui}.

The results from the user testing are shown in table \ref{tab:guifeedback}. In addition to rate the different suggestions, the test persons got to comment them as well.

\begin{table}[ht]
\begin{center}
\begin{tabular}{|l|l|l|} 
\hline
Suggestion  &   Comments & Rating \\
 \hline
1           &   - A solution using panes to represent each robots gets complex if & 2\\
            &   many robots are connected. & \\
            &   - It is cumbersome to have to switch between panes to get info &\\
            &   about different robots. & \\
\hline
2           &   - The main screen presents almost too much information.  & 3\\
            &   - Choosing modes before entering the main screen can help &\\
            &   prevent user errors.  &\\
\hline
3           &   - It is straightforward to present the information about the & 1\\
            &   respective robots in a separate screen.  &\\
            &   - Less is more.  &\\
\hline
\end{tabular}
\end{center}
\caption{User feedback regarding GUI Design}
\label{tab:guifeedback}
\end{table}


\subsection{Select plan}
\label{sec:selectplan}
This phase focuses on choosing a plan for the design, based on the results from the ``Evaluation and feedback'' phase. The group discussed the results and evaluated the suggestions based on their experience and selected the plan.

The design that was chosen consists of functionality and design parts from all of the three suggestions. The mode selection from suggestion 1 and 2, see Figure \ref{fig:gui2}, was included to prevent user errors that may have occurred if the selection was present in the main window. The three window solution in suggestion 3 laid the basis for the design of the main windows in the application. Some minor changes, such as moving the ``Start'' and ``Stop'' buttons and removing the ``Local map'' as well as adding a menu bar at the top of the main window were made. 
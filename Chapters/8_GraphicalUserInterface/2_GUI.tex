\section{Programming the Graphical User Interface}
The \acrshort{gui} was programmed and designed in Netbeans \ref{sec:software}, since Netbeans offers a very intuative \acrshort{gui} builder for Java applications. Netbeans generates a .form file in additional to the normal .java file, the .form file is encoded in xml and contains all parameters the \acrshort{gui} builder needs. When designing a gui in netbeans the user do not need to worry about the .form file, as Netbeans does all the work. Netbeans will automaticly recogninse a Netbeans project and thne load .java and .form files together, because of this different users can easily modify the existing project.

The \acrshort{gui} is initialised by calling \textit{new MainGUI(this)} from the class Applicaiton, and MainGUI has the responsibility of initialising the rest of the \acrshort{gui}. There is a total of 9 java classes in the gui package, where 8 of them are extending JFrame and one extending JPanel.
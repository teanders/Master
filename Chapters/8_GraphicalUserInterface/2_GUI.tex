\section{Programming the Graphical User Interface}
\label{secr:proggui}
The \acrshort{gui} was programmed and designed in Netbeans (Section \ref{sec:software}), since Netbeans offers a very intuitive \acrshort{gui} builder for Java applications. Netbeans generates a ``.form''-file in addition to the normal ``.java''file, and the ``.form''-file is encoded in XML which contains all parameters the \acrshort{gui}-builder needs. When designing a GUI in Netbeans the user do not need to worry about the ``.form''-file, as Netbeans does all the work. Netbeans will automatically recognise a Netbeans project and then load ``.java''- and ``.form''-files together, because of this different users can easily modify an existing project.

The \acrshort{gui} is initialised by calling \textit{new MainGUI(this)} from the Application class, and MainGUI has the responsibility of initializing the rest of the \acrshort{gui}. There is a total of 9 java classes in the GUI package, where 8 of them are extending JFrame and one extending JPanel. Detailed information about the classes can be found in the Javadoc, see Appendix \ref{app:javadoc}.
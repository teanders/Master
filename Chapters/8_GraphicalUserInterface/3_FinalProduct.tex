\section{The Final Product}
The final product can be viewed in Figure \ref{img:mainWindowFinal}, which shows the main window. In this figure, there are two simulated robots (Arduino and Atmel) driving around in a simulated world. The GUI that controls the simulation can be opened by pressing the menu Window and then pressing Simulator. For more information about the Simulator GUI, please refer to Thon (2016).

\mabox{1}{Results/guiFinal.png}{GUI, Main window}{img:mainWindowFinal} 

The main \acrshort{gui} is divided into seven parts as illustrated by Figure \ref{img:mainWindowFinal}. Objects that have similar functions are grouped together, and the different groups are separated by borders. When hovering the mouse over an object in the \acrshort{gui}, the info panel marked with seven in Figure \ref{img:mainWindowFinal} will give extra information about that object, in Figure \ref{img:mainWindowFinal} the user has the mouse hovering the map. The \acrshort{gui} is created with the Gestalt Principles (Section \ref{sec:gestaltprinciples}) and Schneidermans 8 golden rules (Section \ref{sec:schneiderman}) in mind and if anything is unclear the user can press the menu item Help and then User manual, which will open the user manual in the default PDF viewer. The user manual can be found in Appendix \ref{app:userman}, and it provides information about every function in the user interface.

Figure \ref{fig:guiflow} illustrates the different parts of the final \acrshort{gui}. After choosing mode, the mode-specific window opens. Clicking ``Connect robots'' brings up the related window, and as the scan finishes, the available robots gets listed. After selecting the desired robot(s) and clicking ``Connect'' the communication with the robot(s) establishes, and the server receives a handhake-message from the robot(s) confirming the connection. Furthermore, the user sets the initial values for the specific robots, and after the values are set, clicking ``Start'' will initialise the mapping.

\mabox{.8}{Results/GUIFlow.png}{GUI Flow}{fig:guiflow}
\chapter{Conclusion}

\textbf{Konkluder med å besvare de følgende spørsmålene:}
\begin{itemize}
    \item Svarer resultatet til det som først var planlagt?
    \item Fungerer kommunikasjonen godt nok? Er vi fornøyde med den?
    \item Er roboten et ferdig produkt? Tilfører den funksjonalitet som de andre robotene ikke har? Er vi fornøyde med den?
    \item Er GUI godt designet? Er det brukervennlig, men likevel avansert nok? Kunne noe vært annerledes?
    \item Er applikasjonen et godt produkt? Har vi klart å nyskape den, men likevel tatt vare på tidligere funksjonalitet?
    \item Har prosjektfasen vært lærerik? Har vi holdt oss til planen? Har vi lagt bak skjema? Har vi måttet ta snarveier? Er vi fornøyd?
\end{itemize}

During the Master Thesis, the group has developed a server application and a \acrlong{gui} in Java, software for an nRF51-dongle, as well as building an Arduino Robot. The different parts of the project seem to be a good solution to the problem description the group made prior to the project start.

Choosing Java as programming platform facilitated a clear structure of the code and provided functionality for dividing the system into modules. The server application is developed in terms of the guidelines in ``Code Complete'', and is structured in a way that makes it easy for future projects to understand and further develop the software by adding new functionality. The code is generalised, so that it is possible to add robots with different specifications than the ones already existing, e.g. a robot with more than four \acrlong{ir} sensors.

The user interface is developed during a design process where the human perception has been in focus. The process has helped the group focus on the aspects that are important when designing a \acrshort{gui}, and the final product inherits most of the functionality provided in the previous solutions but presents them in a more user-friendly way. It can be said to be in keeping with a natural and good user experience.


%Fordeler/ulemper? Fordeler/ulemper med java (valg av programming platform)? Responsivt og kjapt? Strukturert og oversiktlig? Lett å vedlikeholde? lett å utvide? lett å endre en modul? lett å legge til andre typer roboter?
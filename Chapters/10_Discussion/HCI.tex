%Utforming av brukergrensesnittet, hvorfor designfase var viktig, hvilke prinsipp og regler har lagt til grunne for designvalg?


%Fungerer det? Tilstrekkelig med funksjonalitet? Oversiktlig og brukervennlig? Tillater funksjonalitet både for nybegynner og rutinert bruker?


%Nytt GUI vs gammelt GUI:
%- funksjonene i det gamle er også tilstede i det nye
%- ryddigere og mer oversiktlig
%- mer intuitivt
\section{Graphical User Interface}
The design process is a vital part of developing a system that uses \acrfull{hci} and therefore has the group let the design process form the basis for the developing of the \acrlong{gui}. Through the process of designing the \acrshort{gui}, the group has defined problems, explored alternatives and found a solution that the group believes is the best solution to the initial design problem, see Section \ref{sec:sensethegap} and \ref{sec:understprob}.

Especially has the Gestalt Principles, Section \ref{sec:gestaltprinciples}, and the eight golden rules of Schneiderman, Section \ref{sec:schneiderman}, been in focus during the layout of the design. With the Gestalt Principle ``Proximity'' in mind is the GUI designed in a way that objects that are connected are placed next to each other, as well as the objects are linked to common functions. Figure \ref{fig:startstop} shows the ``Start - Stop'' panel from the final product. 


\mabox{.3}{Discussion/StartStopPanel.png}{Start-Stop Panel from GUI, Main Window}{fig:startstop}

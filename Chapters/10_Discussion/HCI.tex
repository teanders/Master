%Utforming av brukergrensesnittet, hvorfor designfase var viktig, hvilke prinsipp og regler har lagt til grunne for designvalg?


%Fungerer det? Tilstrekkelig med funksjonalitet? Oversiktlig og brukervennlig? Tillater funksjonalitet både for nybegynner og rutinert bruker?


%Nytt GUI vs gammelt GUI:
%- funksjonene i det gamle er også tilstede i det nye
%- ryddigere og mer oversiktlig
%- mer intuitivt
\section{Graphical User Interface}
The design process is a vital part of developing a system that uses \acrfull{hci} and therefore has the group let the design process form the basis for the developing of the \acrlong{gui}. Through the process of designing the \acrshort{gui}, the group has defined problems, explored alternatives and found a solution that the group believes is the best solution to the initial design problem, see Section \ref{sec:sensethegap} and \ref{sec:understprob}.

Especially has the Gestalt Principles, Section \ref{sec:gestaltprinciples}, and the eight golden rules of Schneiderman, Section \ref{sec:schneiderman}, been in focus during the layout of the design. With the Gestalt Principle ``Proximity'' in mind is the \acrshort{gui} designed in a way that objects that are connected are placed next to each other, as well as the objects are linked to the same functions. Figure \ref{fig:startstop} shows the ``Start - Stop'' panel from the final product. The panel demonstrates how proximity is used as a tool to connect functionality of two buttons as well as an indicator lamp and a descriptive text. According to the principle ``Common Fate'', all of the objects that are connected are affected when interacting with one of them. Let's say the stop button is clicked: The clicked button will be disabled, the ``Start'' button will be enabled, the indicator lamp will turn red as well as the descriptive text will change to ``Program not running''.

\mabox{.3}{Discussion/StartStopPanel.png}{Start-Stop Panel from GUI, Main Window}{fig:startstop}

The layout of the main window is similar whether the user chooses to run the ``Simulator'' or the ``Real World'' version of the application. This is a choice made taking Schneiderman's rule about ``Consistency'' into account. The user feels that the user interface is complete and recognisable and this leads to better usability. According to Schneiderman's rule number 3, all actions leads to something in the \acrshort{gui}. As tasks are being done the application responds by showing progress bars and information in the ``Information Panel'', see panel 7 in Figure \ref{img:mainWindowFinal}. 

As the examples above illustrates, the \acrshort{gui} follows different rules and principles making the interface more usable for both experienced and new users.
\section{Arduino Robot}
The Arduino robot is built with the same specs as the AVR robot; it has the same wheels, gears, \acrshort{ir} sensors and tower, compass, gyroscope and the same charging mechanism. What differs from the AVR robot is the wheel encoders, the motor controller, the motors, the ability to turn of charging towers and of course the Arduino micro controller with the selfmade \acrshort{pcb} shield. 

Using an Arduino as microcomputer has several benefits, it is easy to connect other components to its circuit, and by creating a shield it gets even better. This makes it also great if the robot should get new sensors in the years to come. It is also straightforward to upload C code by following the steps in Section \ref{sec:configSoftare}. 

Arduino has a 5V output, but this was not enough to drive the whole circuit, the group had to use the 5V from the motor controller instead. The motor controller can deliver up to 2A at 5V in additional to the motors and has no problem with running the robot.

At this point everything on the Arduino Robot works as intended, but one wheel encoder got damaged during last week of the project period. It is very important that the wheel encoder is placed 2-3mm away from the magnet and that it is not moved directly in touch with the magnet wheel. Due to the wheel encoder not working the robot can not estimate its position, the robot has a compass, gyroscope and accelerometer all working, but the software is not optimised around them. For more information about the software see Ese (2016).
\section{Arduino-Robot}
The Arduino-robot is built with the same specifications as the AVR-robot; it has the same wheels, gears, \acrshort{ir} sensors, compass, gyroscope and the same charging mechanism. What differs from the AVR-robot is the wheel encoders, the motor controller, the motors, the ability to turn off the charging towers and the Arduino microcontroller with the self-made \acrshort{pcb} shield. 

Using an Arduino as microcomputer has several benefits, it is easy to connect other components to its circuit, and by creating a shield it gets even better. This makes the robot easy to modify if the robot is to be extended, for example, with new sensors in the years to come. It is also straightforward to upload C code by following the steps in Section \ref{sec:configSoftware}. 

Arduino has a 5V output, but as the output could not provide power to the whole circuit, the group had to use the 5V from the motor controller instead. The motor controller can deliver up to 2A at 5V in addition to the motors, and it has no problem powering the robot.

At this point, everything but one wheel encoder works on the Arduino-robot. The wheel encoder got damaged during the last week of the project period. It is very important that the wheel encoder is placed 2-3 mm away from the magnet, and that it does not touch the magnet wheel. Due to the wheel encoder not working, the robot can not estimate its position. The robot has a compass, gyroscope and accelerometer all working, but the software is not optimised to use input from them. For more information about the software, see Ese (2016).
%Fordeler/ulemper? Fordeler/ulemper med java (valg av programming platform)? Responsivt og kjapt? Strukturert og oversiktlig? Lett å vedlikeholde? lett å utvide? lett å endre en modul? lett å legge til andre typer roboter?
\section{Selecting Programming Platform}
The group had to choose programming language early in the project period. The members had experience with several languages, such as Java, C, Google Go and Python, from earlier school projects.  Choosing a suitable programming platform to develop the system was vital, and the group had to take several factors into account when selecting one to go forth with.

The existing system was written using MATLAB and Simulink. The program consisted of several ``.m''-files and included different toolboxes,  such as CAS \footnote{CAS Robot Navigation Toolbox} for  implementation of \acrshort{slam} functionality. In addition to requiring a bit of processing power and memory from the computer, MATLAB is not suited for developing software that should provide \acrshort{gui} and real-time functionality.

Java, as an object-oriented language, had many features that suited the system development. Using packages and classes makes it easy to divide the system into modules with different responsibilities, and this allows for a clear and neat structure of the application. As software developed in Java is build and compiled before running, as well as being able to be run on every computer that has the \acrfull{jvm}, the software program is cross-platform and lightweight.
\section{Background information}
\begin{itemize}
    \item rescue
    \begin{itemize}
        \item earthquake
        \item radioactive areas
    \end{itemize}
    \item security
    \item industry
    \begin{itemize}
        \item inspection of factories
        \item 
    \end{itemize}
    \item military
    \begin{itemize}
        \item demining
        \item battlefields
    \end{itemize}
    \item exploration of unknown environments
    \item swarming
    \begin{itemize}
        \item time saving
        \item doesn't need advanced/complex robots
        \item redundant
        \item modular, easy to expand
        \item verify results
    \end{itemize}
    \item Animal Kingdom
    \begin{itemize}
        \item Ants
        \item Termites
        \item Bees
    \end{itemize}
\end{itemize}

Selvnavigerende autonome roboter kan benyttes i svært mange ulike bruksområder.

\subsubsection{Search \& Rescue}
In situations where massive destructions has happened due to natural disasters, such as earthquakes, it is very crucial to track survivors soon after the incident. Using the traditional probes and boroscopes rescue teams can't look further than 18-20 feet into a destroyed building. Modern technology allows search \& rescue teams to send in teams of cooperating autonomous robots that can localize the survivors and let the rescue team understand the surroundings based its recordings (http://crasar.org/) (http://www.popsci.com/technology/article/2011-03/six-robots-could-shape-future-earthquake-search-and-rescue). 
But the use of robots in search \& rescue is not limited to natural disasters, but it can also be applicable in man made situations such as warfare areas and factory explosions (Chernobyl).

% http://www.allonrobots.com/rescue-robots.html
% http://www.ssrr-conference.org/2013/files/pdf/SSRR-2013-keynote_Simon_Lacroix.pdf
% http://spectrum.ieee.org/automaton/robotics/industrial-robots/japan-earthquake-robots-help-search-for-survivors
% http://ieeexplore.ieee.org/stamp/stamp.jsp?arnumber=6696431

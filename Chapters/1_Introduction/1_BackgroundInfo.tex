\section{Background Information}
It has always been important for human to record their surroundings, and the history of mapping can be traced back to more than 3000 years B.C. The first maps consisted of trade routes, hunting grounds and small areas that included places of interest. The maps were not very accurate, with bad relationships and few details included, but the maker was able to show the features that the maker wished to record \cite{histmap}.

The Greeks and the Romans redefined the art of mapping, and when Claudius Ptolemaeus published his work ``Geographia'' in year 150 A.D, the European geographic thinking was revolutionised \cite{histmap}. As the years have passed, modern technology has expanded the ways people can map areas, from sundials and compass to the present GPS.

Today's technology enables humans to automate the way areas are mapped, using swarms of cooperating robots that use sensors to ``perceive'' their surroundings. In situations where massive destructions have happened due to natural disasters, it is crucial to track survivors soon after the incident. Rescue teams cannot look further than 18-20 feet into a destroyed building using traditional probes and boroscopes \cite{crasar}, but modern technology allows search \& rescue teams to send in several cooperating autonomous robots that can localise the survivors, and let the rescue team understand the areas based on the robot's recordings \cite{popsci}. The use of robots in search \& rescue is not limited to natural disasters, but can also be applicable in man-made situations such as warfare areas and factory explosions.

Mapping unknown areas using cooperating robots has been studied and implemented through projects at NTNU during the last 12 years. In some projects the students have built robots that use different kinds of sensors, while others have designed/improved server software that exploits the data gathered by the robots. An important aspect of the projects has been to exploit the functionality of cheap parts to do advanced tasks, and this has also been the focus of this thesis.
\section{Previous work}
When this thesis was started, a server control system developed in MATLAB existed. The server could connect to one robot with the use of a Bluetooth-dongle. If the user wanted to connect to several robots, several instances of the MATLAB server application had to be started. This meant that one instance of MATLAB per robot was needed. The different MATLAB instances, that ran on the same computer, communicated with each other via TCP/IP. The approach required a lot of computer resources and scaled badly when more than one robot was connected.

The MATLAB software relied on a third-party toolbox ``Centre for Autonomous System Toolbox'' first introduced in a project done by Syvertsen in 2005, and in the following years the server software and the robots were extended and modified several times. Please refer to Homestad (2013) for an extended elaboration about the content of the earlier reports.

During the project and master thesis of Halvorsen (2013 \& 2014) cooperation between robots were introduced. The work with the NXT-robot, server application and the implemented simulator was continued.

The lack of the real-time aspect in the system laid the foundation for Ese (2015). During his project, the work with a real-time \acrshort{os} was started. In addition, the NXT-robot was re-built such that the sensors mounted on the robot were the same as on the AVR-robot, but the NXT-robot was not able to map autonomously. It could only produce simple maps, and the reason behind this was not investigated. The developed real-time \acrshort{os} was implemented on the AVR-robot during the project. Ese described the current state of the server application as full of bugs. The simulator worked but crashed sporadic, the application printed error messages during start-up leading to forced restart of the software, the collaboration did not work, and the \acrshort{gui} presented buttons that did not have any function.
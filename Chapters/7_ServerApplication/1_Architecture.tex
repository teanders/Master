\section{Designing the System Architecture}
To start off the system design, the group acquired the reports and code from the previous projects. By reading the code, testing the functionality of the \acrshort{gui} and studying the reports the group got an overview and an understanding of the existing system. It was important to know the current system before starting to develop the new system, and therefore, the group used a good deal of time to study it.

Initially, the group defined the scope of the new system. The system should:
\begin{itemize}
    \item inherit the functionality from the existing system.
    \item be stable and not introduce bugs.
    \item present mapped data in an easily understandable way.
    \item provide a user-friendly \acrshort{gui}.
    \item be easy to understand and improve in future projects.
\end{itemize}
The list of system goals laid a basis for the system named ``System for Self-Navigating Autonomous Robots'', henceforth referred to as \acrshort{ssnar}.

One of the primary goals was to develop the system in a way that makes it easy to understand for groups that want to improve the system in future projects. The group had modular programming in mind in every part of the system design process to achieve this goal, and therefore, the system was divided into subsystems and packages that took care of different areas of responsibility and functionality. Figure \ref{fig:brainstorming} shows how the group was thinking during the brainstorming, trying to structure the responsibilities into modules. 

Subsequently, classes were defined in each module. According to the naming conventions in ``Code Complete'' \cite{stevemcconnell2004}, see Section \ref{sec:nameconv}, the classes was named according to their responsibilities with self-explanatory names, making them easy to understand without having to read their code. Furthermore, the classes routines were defined. All of the routines were designed to do one thing and one thing only, for facilitating high cohesion and low coupling, as described in Section \ref{sec:softconstruct}.

Throughout the developing, the group held coordination meetings with both Erlend Ese and Eirik Thon to make sure the system was designed in the best way possible.
\mabox{.5}{Methods/Brainstorming.JPG}{Dividing the system into subsystems and packages}{fig:brainstorming}
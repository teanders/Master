\section{Designing the system architecture}
\begin{itemize}
	\item Hvordan satte vi igang arbeidet? Vi fulgte ``The five levels of design'':
	\begin{itemize}
		\item Studerte tidligere arbeid
		\item Gikk systematisk gjennom matlab-koden og noterte oss funksjonalitet
		\item Satte opp ansvarsområder
		\item Lagde tankekart for ansvarsområder med tilhørende klassedefinisjoner
		\item Funksjonalitet for klassene
		\item 
	\end{itemize} 
\end{itemize}

To start off the system design, the group acquired the reports and code from the previous projects. By reading the code, testing the functionality of the \acrshort{gui} and studying the reports the group got an overview and an understanding of the existing system. It was important to know the existing system before starting to develop the new system, and therefore the group used a good deal of time to study it.

Initially, the group defined the scope of the new system. The system should:
\begin{itemize}
	\item inherit the functionality from the existing system.
	\item be stable and not introduce bugs.
	\item present mapped data in an easily understandable way.
	\item provide a user friendly \acrshort{gui}.
	\item be easy to understand and improve in future projects.
\end{itemize}
The list of system goals laid a basis for the ``System for Self-Navigating Autonomous Robots'', henceforth referred to as \acrshort{ssnar}.

One of the most important goals was to develop the system in a way that makes it easy to understand for groups that want to improve the system in future projects. To achieve this, the group had modular programming in mind in every part of the system design process, and therefore the system was divided into subsystems and packages that took care of different areas of responsibility and functionality.
\newpage
\section{Limitations}
\label{secr:limitations}
\minisection{Disconnecting before connecting}
Due to an indexing problem in the nRF51-dongle, it was needed to constrain the progress of connecting to additional robots. Before initializing a scan for new robots, all of the robots that are attached to the system has to be disconnected. The reason is that when the nRF51 scans and discovers new unattached robots, it indexes them from 0 and up, even though there already were robots connected from before the scan. This will lead to several robots with the same ID (e.g. two robots with ID = 1), which is undesirable.

\minisection{Connecting to more than three robots in real world, and ten in the simulated world}
The server application is limited to a maximum of ten connected robots. As explained in Section \ref{sec:devnrf}, the nRF51-dongle has a limitation of maximum three peripherals connected at the same time. This results in a maximum of three robots during real world mode and maximum ten robots during simulated world mode. It is easy to change the limit in the application, but as of right now it is limited to checking one digit in the ID parameter of the messages, as the nRF51 only sends one digit IDs.
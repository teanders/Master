\section{System Flow}
\subsubsection{Application}
\mabox{.6}{Results/UML-Package-Application}{UML of ``Application''-package}{fig:umlapplication}
\task{For further details about the routines within the classes, see APPENDIX JAVADOC}
Figure \ref{fig:umlapplication} shows the \acrfull{uml} diagram of the ``Application'' package. As seen in the figure, the package consists of the three classes ``Application'', ``RobotController'' and ``Installer''.

\textit{Installer}

The ``Installer'' class' static method ``generateSystemDependantLibraries'' generates and moves libraries that are necessary for the computer to run the application to the root folder of the application. Since the software uses serial communication, some library files (``.dll''(Windows), ``.jnilib''(Mac OS)) has to be present to for instance be able to open and close COM-ports. 


\mabox{.6}{Results/UML-Package-Communication}{UML of ``Communication''-package}{fig:umlcommunication}
\mabox{.6}{Results/UML-Package-Robot}{UML of ``Robot''-package}{fig:umlrobot}
\mabox{.6}{Results/UML-Package-GUI}{UML of ``GUI''-package}{fig:umlgui}
\mabox{1}{Results/UML}{UML of ``SSNAR''}{fig:uml}